\documentclass[]{elsarticle} %review=doublespace preprint=single 5p=2 column
%%% Begin My package additions %%%%%%%%%%%%%%%%%%%

\usepackage[hyphens]{url}

  \journal{SoftwareX} % Sets Journal name

\usepackage{lineno} % add
  \linenumbers % turns line numbering on

\usepackage{graphicx}
%%%%%%%%%%%%%%%% end my additions to header

\usepackage[T1]{fontenc}
\usepackage{lmodern}
\usepackage{amssymb,amsmath}
\usepackage{ifxetex,ifluatex}
\usepackage{fixltx2e} % provides \textsubscript
% use upquote if available, for straight quotes in verbatim environments
\IfFileExists{upquote.sty}{\usepackage{upquote}}{}
\ifnum 0\ifxetex 1\fi\ifluatex 1\fi=0 % if pdftex
  \usepackage[utf8]{inputenc}
\else % if luatex or xelatex
  \usepackage{fontspec}
  \ifxetex
    \usepackage{xltxtra,xunicode}
  \fi
  \defaultfontfeatures{Mapping=tex-text,Scale=MatchLowercase}
  \newcommand{\euro}{€}
\fi
% use microtype if available
\IfFileExists{microtype.sty}{\usepackage{microtype}}{}
\usepackage[]{natbib}
\bibliographystyle{plainnat}

\ifxetex
  \usepackage[setpagesize=false, % page size defined by xetex
              unicode=false, % unicode breaks when used with xetex
              xetex]{hyperref}
\else
  \usepackage[unicode=true]{hyperref}
\fi
\hypersetup{breaklinks=true,
            bookmarks=true,
            pdfauthor={},
            pdftitle={Microxanox: an R package for simulating an aquatic MICRobial ecosystem that can occupy OXic or ANOXic states.},
            colorlinks=true,
            urlcolor=blue,
            linkcolor=blue,
            pdfborder={0 0 0}}

\setcounter{secnumdepth}{5}
% Pandoc toggle for numbering sections (defaults to be off)


% tightlist command for lists without linebreak
\providecommand{\tightlist}{%
  \setlength{\itemsep}{0pt}\setlength{\parskip}{0pt}}

% From pandoc table feature
\usepackage{longtable,booktabs,array}
\usepackage{calc} % for calculating minipage widths
% Correct order of tables after \paragraph or \subparagraph
\usepackage{etoolbox}
\makeatletter
\patchcmd\longtable{\par}{\if@noskipsec\mbox{}\fi\par}{}{}
\makeatother
% Allow footnotes in longtable head/foot
\IfFileExists{footnotehyper.sty}{\usepackage{footnotehyper}}{\usepackage{footnote}}
\makesavenoteenv{longtable}


%% \documentclass[preprint,12pt, a4paper]{elsarticle}

%% Use the option review to obtain double line spacing
%% \documentclass[authoryear,preprint,review,12pt]{elsarticle}

%% For including figures, graphicx.sty has been loaded in
%% elsarticle.cls. If you prefer to use the old commands
%% please give \usepackage{epsfig}

%% The amssymb package provides various useful mathematical symbols
\usepackage{amssymb}
%% The amsthm package provides extended theorem environments
%% \usepackage{amsthm}

%% The lineno packages adds line numbers. Start line numbering with
%% \begin{linenumbers}, end it with \end{linenumbers}. Or switch it on
%% for the whole article with \linenumbers.
\usepackage{lineno}

\usepackage{float}
\restylefloat{table}

\journal{SoftwareX}

%\usepackage{lineno}
%\linenumbers
%\usepackage{setspace}
%\doublespacing
%\usepackage{fancyhdr}
%\fancyhead{}
%\fancyfoot{}
%\fancyhead[CO,CE]{Microxanox R package}
%\pagestyle{fancy}
%



\begin{document}


\begin{frontmatter}

  \title{Microxanox: an R package for simulating an aquatic
\(MICR\)obial ecosystem that can occupy \(OX\)ic or \(ANOX\)ic states.}
    \author[University of Zürich]{Rainer M Krug%
  %
  \fnref{Corresponding Author}}
   \ead{Rainer.Krug@uzh.ch; Rainer@krugs.de} 
    \author[University of Zürich]{Owen L. Petchey}
   \ead{Owen.Petchey@ieu.uzh.ch} 
      \affiliation[University of Zürich]{Department of Evolutionary
Biology and Environmental Studies, Winterthurerstrasse 190, 8057 Zurich}
    \cortext[cor1]{Corresponding author}
    \fntext[1]{Corresponding Author}
    \fntext[2]{Equal contribution}
  
  \begin{abstract}
  Ca. 100 words.
  \end{abstract}
    \begin{keyword}
    reproducibility; regime shift; \sep 
    reproducibility; regime shift;
  \end{keyword}
  
 \end{frontmatter}

\hypertarget{section}{%
\section{\texorpdfstring{\textcolor{red}{TODO}}{}}\label{section}}

\begin{itemize}
\tightlist
\item
  \textbf{\textcolor{red}{update links}}
\item
  \textbf{\textcolor{red}{update cross references}}
\item
  \textbf{\textcolor{red}{check keywords}}
\item
  \textbf{\textcolor{red}{add more references}}
\item
  \textbf{\textcolor{red}{expand on Impact and Conclusion section}}
  \pagebreak
\end{itemize}

\hypertarget{required-metadata}{%
\section{Required Metadata}\label{required-metadata}}

\hypertarget{current-code-version}{%
\subsection{Current code version}\label{current-code-version}}

Ancillary data table required for subversion of the codebase.

\begin{longtable}[]{@{}
  >{\centering\arraybackslash}p{(\columnwidth - 4\tabcolsep) * \real{0.0483}}
  >{\raggedright\arraybackslash}p{(\columnwidth - 4\tabcolsep) * \real{0.4138}}
  >{\raggedright\arraybackslash}p{(\columnwidth - 4\tabcolsep) * \real{0.5379}}@{}}
\toprule
\begin{minipage}[b]{\linewidth}\centering
\textbf{Nr.}
\end{minipage} & \begin{minipage}[b]{\linewidth}\raggedright
\textbf{Code metadata description}
\end{minipage} & \begin{minipage}[b]{\linewidth}\raggedright
\textbf{Please fill in this column}
\end{minipage} \\
\midrule
\endhead
C1 & Current code version & v0.9.0 \\
C2 & Permanent link to code/repository used for this code version &
\url{https://github.com/UZH-PEG/microxanox} \\
C3 & Code Ocean compute capsule & \\
C4 & Legal Code License & CC BY 4.0 \\
C5 & Code versioning system used & git \\
C6 & Software code languages, tools, and services used &
\href{https://cran.r-project.org/index.html}{R} \\
C7 & Compilation requirements, operating environments &
\href{https://cran.r-project.org/index.html}{R (\textgreater= 4.1.0)} \\
& & magrittr \\
& & tibble \\
& & ggplot2 \\
& & patchwork \\
& & grDevices \\
& & stats \\
& & mgcv \\
& & deSolve \\
& & dplyr \\
& & tidyr \\
& & stringr \\
& & multidplyr \\
& & \\
& & \\
& & \\
& & \\
& & \\
C8 & If available Link to developer documentation/manual &
\textcolor{red}{TO BE ADDED} \\
C9 & Support email for questions & Rainer.Krug@uzh.ch;
Rainer@krugs.de \\
\bottomrule
\end{longtable}

\hypertarget{motivation-and-significance}{%
\section{Motivation and
significance}\label{motivation-and-significance}}

Mathematical models play a key role in the development of understanding
about how ecosystems work and how they respond to environmental changes
\citep{Binzer2016a, Chaparro-Pedraza2021, Vasseur2005}. They are also
critical for developing hypotheses that can then be tested in
experimental studies. One area of ecology in which simple models have
played a very influential role is the area of how ecosystems respond to
gradual environmental change \citep{REF_NEEDED}. - Scheffer, M.,
Carpenter, S., Foley, J. a, Folke, C. \& Walker, B. (2001). Catastrophic
shifts in ecosystems. Nature, 413, 591--6.

It is conceivable that an ecosystem state, such as the total biomass of
a particular type of bacteria, may remain unchanged when something about
its environment changes, such as the temperature. It is possible that
the ecosystem state changes gradually itself. It is also possible that
the ecosystem state may change abruptly into a new state that is
difficult to recover from \citep{Scheffer2001}.

This possibility for abrupt, perhaps catastrophic changes that are
difficult to reverse causes considerable concern
\citep{Collins2021, Northrop2021, Vandermeer2019}.

An example where the gradual change of a single environmental variable
is causing an abrupt change of the system is the switch from an aerobic
(oxygen is available for metabolism) to anaerobic (oxygen generally
unavailable) system. This system has been investigated by
\citet{Bush2017} in a simulation study of a mathematical model of the
system. They showed that gradual change in the environment could cause
catastrophic changes in the ecosystem state that would be difficult to
reverse.

One limitation of the study by \citet{Bush2017} was limited biodiversity
in the ecosystem model. Specifically, there were three functional groups
of bacteria, but within each of these groups, there was no biodiversity.
This then leaves open the question of if and how biodiversity within
functional groups, in their model, affects the ecosystem response to
environmental change. This limitation is not limited to the study of
\citet{Bush2017}, there are few if any studies of the effects of
biodiversity on abrupt transitions between ecosystem states.

We decided to fill this research gap, by making a simulation study of
how within functional group biodiversity affects ecosystem responses to
environmental change \citet{Limberger2022}, and to base our work on the
work and model of \citet{Bush2017}. It was with this goal in mind that
we developed the \emph{microxanox} package. The first stage of
development was to write code from scratch (as there was no available
code to start from) and to confirm that this new implementation would
reproduce the previously published results. The resulting reproduction
is available as one of the package vignettes:
\href{LINK\%20NEEDED}{vignette Partial reproduction of Bush et al}.

The second stage was to add functionality that would be necessary to
answer our research question. Most importantly, we made it possible to
have multiple species of bacteria within each of the three functional
groups, for the multiple species to differ in their characteristics, and
to vary the number of species and amount of variability among them. We
also added functionality that allowed: temporally varying environmental
conditions, addition of random noise to state variables, and
immigration. In addition to the model itself, the package includes some
functions to analyse the results as well as visualize the results to
provide a starting point for customized visualizations based on own
requirements. The basic and additional functionality is described in the
package \href{LINK_NEEDED}{User Guide}.

\hypertarget{software-description}{%
\section{Software description}\label{software-description}}

\begin{quote}
\{\textgreater\textgreater Describe the software in as much as is
necessary to establish a vocabulary needed to explain its
impact.\textless\textless\}
\end{quote}

The \emph{microxanox} package is for simulating a three functional group
system (\emph{CB}: cyanobacteria, \emph{PB}: phototrophic sulfur
bacteria, and \emph{SB}: sulfate-reducing bacteria) with four chemical
substrates (\emph{P}: phosphorus, \emph{O}: oxygen, \emph{SR}: reduced
sulfur, \emph{SO}: oxidized sulfur). It includes feedback between
biogeochemical processes and is based on \citet{Bush2017} (See
\citet{Bush2017} for a detailed discussion of the model). At the core of
the simulations is a set of ordinary differential equations (specified
in the function \texttt{bushplus\_dynamic\_model()}, though users do not
need to call this function directly). There are functions for running
individual simulations and for running a set of simulations across, for
example, a range of environmental conditions.

To make the simulation run with multiple species, we expressed different
species characteristics in the elements of vectors and matrices. We also
adapted the code for the ordinary differential equations to include the
vectors and matrices, and to use matrix mathematics. In this way, we
made it possible to run simulations with different numbers of species
without having to change the underlying code.

The package functions and code are structured in modular, so that some
new functionality can be easily added. E.g. the event definition can
easily be changed, or other aspects can be adjusted. All values in the
parameter object can be changed as needed and the general structure of
the code should make it not to difficult to adapt the model to other
similar systems.

\hypertarget{software-architecture}{%
\subsection{Software architecture}\label{software-architecture}}

\begin{quote}
\{\textgreater\textgreater Give a short overview of the overall software
architecture; provide a pictorial component overview or similar (if
possible). If necessary provide implementation
details.\textless\textless\}
\end{quote}

The framework used when writing this package aimed to maximise
simplicity for the user, and to make it easy to reproduce results (see
the supplement to \citet{Limberger2022} for an example of how this is
used). As such, all the parameters needed to run a simulation or find a
stable state are contained in a single object (which can easily be
created using included functions). This parameter object is given to a
function that runs the simulations and returns the results. The returned
results object is identical to the parameter object but with an
additional slot named \texttt{results}, which contains the results of
the run. Thus the returned results object contains the simulation
conditions (parameters) as well as the rsults, and can be used to run
the simulation again. This promotes reproducibility and makes
incremental changes of individual parameters and re-running the
simulations straightforward.

In the following sections we describe how to use the package to run one
simulation and to find steady states across an environmental gradient.

\hypertarget{running-one-simulation}{%
\subsection{Running one simulation}\label{running-one-simulation}}

A typical simulation would look as shown in @ref(fig:runsim\_example).

\begin{figure}

{\centering \includegraphics[width=350px]{./figures/simflow} 

}

\caption{Typical flow of a simulation. Dark Grey boxes: commands necessary for simulation; Light Grey:Saving of parameter and results; Lightest Grey: Different non specified commands.}\label{fig:runsim_example}
\end{figure}

A simulation is run using the \texttt{run\_simulation()} function. In
this function, the ODEs are solved using the function \texttt{ode()} in
the package \emph{deSolve} package \citep{Soetaert2010}. The
\texttt{run\_simulation()} function needs only one argument - an object
as created by the function \texttt{new\_runsim\_parameter()}. The
parameter object returned by \texttt{new\_runsim\_parameter()} contains
among other things the \texttt{strain\_parameter} object, which can be
created by the function \texttt{new\_strain\_parameter()}. For a
detailed description of the parameter objects, their meaning and how
they are created and have values set and changed please see the
\emph{User Guide} which accompanies the package or is available at
\href{@LINK_NEEDED}{User Guide}

After the parameter object has been defined, it can be used in the
\texttt{run\_simulation()} function. The function returns an object
which is identical to the parameter object, except of an additional slot
containing the results. This design produces a fully reproducible object
as it can be used as a parameter object to be fed back into the
\texttt{run\_simulation()} function to run the simulation again from the
parameter used to generate the results before.

\hypertarget{finding-steady-states}{%
\subsection{Finding Steady States}\label{finding-steady-states}}

The general approach used to find the stable state of the system with a
specific parameter set is to run the simulation for a long time and
record the final state. When one does this across a range of
environmental conditions, one discovers how the steady state of the
system responds to the environmental conditions. The package contains
functionality for finding steady states that correspond to values of one
environmental variable, namely the value of oxygen diffusivity.

Two methods for finding steady states are implemented. The first runs a
separate simulation for each combination of starting conditions and
oxygen diffusivity (let us term this the \emph{Replication method}).
This is the method used in the \citet{Bush2017} study. The second runs
only two simulations, with step-wise and slowly temporally increasing
oxygen diffusivity, and the other with step-wise and slowly decreasing
or decreasing oxygen diffusivity. During this temporal environmental
change, the state of the system was recorded just before change to a new
oxygen diffusivity (let us term this the \emph{Temporal method}). We
implemented two methods since there is no definitive best method, and in
order to check if results were sensitive to choice of method.

The replication method is implementad in the function
\texttt{run\_replication\_ssfind()} which takes a parameter object as
returned by the function \texttt{new\_replication\_ssfind\_parameter()}
and the number of cores for multithreading the simulation. As the
multithreading uses the package function \texttt{mclapply()} from the
package \texttt{parallel} \citep{RCoreTeam2022}, the multthreading only
works on Linux and Mac. It is planned to move to \texttt{parLapply()}
\citep{RCoreTeam2022} in a future release.

This temporal method implemented in the function
\texttt{run\_temporal\_ssfind()}, which takes a parameter object as
created by the function \texttt{new\_temporal\_ssfind\_parameter()}. It
is planned for a later release, to run these two simulations in
parallel.

For a more detailed walk-through of these two approaches and explanation
please see the \href{@LINK_NEEDED}{User Guide}.

\hypertarget{visualising-and-analysing-results}{%
\subsection{Visualising and analysing
results}\label{visualising-and-analysing-results}}

From the raw results returned, summary measures about how the ecoystem
stable states respond to environmental change can be extracted. The
function \texttt{get\_stability\_measures()} returns quantities such as
the amount of environmental change required for the system to abruptly
change to a different state.

The function \texttt{plot\_dynamics()} plots a single simulation run, as
returned from the \texttt{run\_simulation()} function. This function is
only provided as a convenience function to provide a way to easily see
the results of a simulation run. An example plot resulting from this
function is shown in @ref(fig:plot-dynamics).

\begin{figure}

{\centering \includegraphics[width=350px]{figures/ug_three_strains_dynamics} 

}

\caption{Plot of results of a simulation run using the function $plot_dynamics()$. Details can be found in the "User Guide" section "Three strains per functional group".}\label{fig:plot-dynamics}
\end{figure}

\hypertarget{impact}{%
\section{Impact}\label{impact}}

\begin{quote}
\{\textgreater\textgreater This is the main section of the article and
the reviewers weight the description here appropriately Indicate in what
way new research questions can be pursued as a result of the software
(if any). Indicate in what way, and to what extent, the pursuit of
existing research questions is improved (if so). Indicate in what way
the software has changed the daily practice of its users (if so).
Indicate how widespread the use of the software is within and outside
the intended user group. Indicate in what way the software is used in
commercial settings and/or how it led to the creation of spin-off
companies (if so).\textless\textless\}
\end{quote}

The free and open source implementation and extension of the model used
in \citet{Bush2017} provides the means of reproducing the results
published while at the same time provides the means of doing unique,
innovative, and important investigations of how ecosystems respond to
environmental change. The design of the package code and functionality
is with reproducibility in mind: the combination of all parameters being
in a single parameter object as well as the return of the simulation as
a result object which inherits from the parameter object provides a
relatively easy to use framework to implement reproducible experiments.

While the package was not intended to provide a code which can be very
easily adapted to other types of ecosystems and types of organism, it
seems likely that the overall framework could be used as a basis for
models of other ecosystems and organisms. For example, all such models
would have parameters that differ among species and need to be described
in an object. All would need to run simulations and sets of simulations
across environmental conditions. Researchers want to model a new
ecosystem do not, therefore, have to start from scratch. Nevertheless,
depending on how different were the new ecosystem to model, the require
code changes could be quite significant.

We evidence the impact of the \emph{microxanox} package by describing
three use cases, each of which was only possible with the package and
that contains new results. The first two use cases are described in
detail in the User Guide and the Partial Reproduction Vignettes. The
third is taken from \citet{REF_NEEDED} for which this R package was
designed. All of these use cases can be expanded to larger numbers of
strains per functional group and variable values van be changed.

\hypertarget{use-case-1-regime-shifts-during-temporal-environmental-change}{%
\subsection{Use case 1: Regime shifts during temporal environmental
change}\label{use-case-1-regime-shifts-during-temporal-environmental-change}}

The package contains functionality to make a specific pattern of
temporal change in the environmental condition oxygen diffusivity; this
functionality forms the basis of the temporal method for finding stable
states. A simple example of this functionality is given in the Partial
Reproduction vignette, which we briefly show here (Figure @ref(fig:uc1).
The example is composed of a single simulation, at the beginning of
which the system is in the oxic state with high abundance of
cyanobacteria. Oyxgen diffusivity is then slowly decreased and
eventually, around hour 30'000 the system switches to the anoxic state,
with high abundance of both sulfur bacteria types. The oxygen
diffusivity is then quickly increased and at around hour 38'000 the
system abruptly switches back to the oxic state.

\begin{figure}

{\centering \includegraphics[width=350px]{./figures/gen_uc1_partrep_temporal_state_switching} 

}

\caption{The temporal dynamics of the ecosystem model when an environmental condition (here parameter *a*, the oxygen diffusivity) changes. Plot of the stable states of the simulation runs under different oxygen diffusivity.}\label{fig:uc1}
\end{figure}

\hypertarget{use-case-2-the-extent-of-hysteresis-depends-on-community-composition}{%
\subsection{Use case 2: The extent of hysteresis depends on community
composition}\label{use-case-2-the-extent-of-hysteresis-depends-on-community-composition}}

The package also contains a function to extract summary features of
ecosystem responses to environmental change, such as the amount of
hysteresis displayed by the ecosystem. Hysteresis is a key feature of
ecosystem responses to environmental change, because it is related to
how difficult it can be to reserve the effects of environmental change
\citep{Scheffer2001}. The amount of hysteresis was measured here as the
extent of the environmental condition for which there were two stable
states. I.e. it was the extent of the environmental conditions for which
historical conditions play an important role in determining the current
system state (a definition of hysteresis). The results show that the
amount of hysteresis depends greatly on the combinations of organisms
present (Figure @ref(fig:uc2)). For example, with only the CB
(cyanobacteria) present, there was no hysteresis. In contrast, the
presence of both CB and SB (sulfate reducing bacteria) led to a large
amount of hysteresis. (These results are also given in the Partial
Reproduction vignette.)

\textbf{\citet{Rainer}\ldots{} the figure needs beautifying please.}
\textbf{\citet{Owen}: is this OK like this?}

\begin{figure}

{\centering \includegraphics[width=350px]{./figures/gen_uc2_user_guide_hysteresis} 

}

\caption{The amount of hysteresis depends on the combination of types of organisms present.}\label{fig:uc2}
\end{figure}

\hypertarget{use-case-3-effects-of-functional-diversity-on-regime-shifts}{%
\subsection{Use case 3: Effects of functional diversity on regime
shifts}\label{use-case-3-effects-of-functional-diversity-on-regime-shifts}}

As discussed in the Introduction section, the package was largely
motivated by the question of how biodiversity influences ecosystem
responses to environmental change. Extensive results concerning this
question are given in a separate publication \citet{Limberger2022}. Here
we describe one of the results, which is that having biodiversity in a
functional group can allow state changes to occur that otherwise would
not have. I.e. biodiversity can qualitatively change the state of the
ecosystem.

Biodiversity was added to the functional groups using the
\texttt{new\_strain\_parameter()} function to create a parameter set
with multiple species per functional group (albeit all with identical
features) and then to add variability among the species by calling the
\texttt{add\_strain\_var()} function. This function takes an already
existing parameter set and adds the specified about of variation. The
new parameter object is then used as before.

Figure @ref(fig:uc3) shows a simulation with two species (strains) in
each of the three functional groups. The ecosystem starts in the oxic
state, though with relatively high abundance of each functional group.
The strain of SB that is more tolerant to oxygen (SB\_1) initially
decreases in abundance, but then increases, and the other (SB\_2) strain
then becomes abundance and SB\_1 declines. Furthermore, the
cyanobacteria crash in abundance, and the system switches to the anoxic
state. Importantly, the switch does not occur if there are two identical
strains with tolerance half way between those in Figure @ref(fig:uc3).

\textbf{Rainer\ldots{} please beautify the figure as required. It comes
from the Limberger et al supplement, Figure 5.4} \textbf{\citet{Owen}:
is this OK like this? Used the graph from the supplement}

\begin{figure}

{\centering \includegraphics[width=350px]{figures/uc3_supplement_5_4} 

}

\caption{The dynamics of the ecosystem when there are two species in each functional group, and some variation (diversity) in species parameters.}\label{fig:uc3}
\end{figure}

\hypertarget{conclusions}{%
\section{Conclusions}\label{conclusions}}

\begin{quote}
\{\textgreater\textgreater Set out the conclusion of this original
software publication.\textless\textless\}
\end{quote}

\emph{microxanox} allows the simulation, visualisation, and analysis of
a model of a microbial ecosystem while allowing variation in the amount
of diversity containing in each of the functional groups of organisms
present. It has been used for the research described in another paper of
ours that is one of the first investigations of the effects of diversity
on ecosystem resilience \citet{Limberger2022}. In that paper, we show
that diversity can have large and important effects of ecosystem
responses, highlighting the need for models such as ours, within which
one can easily manipulate the amount of biodiversity. The
\emph{microxanox} package has also been used to reproduce the results of
the paper that inspired the package development \citet{Bush2017}.

The package greatly lowers the amount of work required in further
investigations of the specific ecosystem modelled. There has, for
example, been quite limited investigation of how biodiversity influences
the short-term responses of the modelled ecosystem to environmental
change. Likewise, the package could be used to power an investigation of
the effects of biodiversity on the usefulness of early warning signals
of abrupt ecosystem change \citep{Scheffer2009}. In addition this
package could be used as a template for the implementation for
developing models of other types of ecosystems and organism. By doing
so, other models can profit from the overall framework used, and the
reproducibility aspects as well as the flexibility implemented.

\hypertarget{conflict-of-interest}{%
\section{Conflict of Interest}\label{conflict-of-interest}}

The authors declare no known conflicting or competing interests
associated with this publication and there has been no significant
financial support for this work that could have influenced its outcome.

\hypertarget{acknowledgements}{%
\section{Acknowledgements}\label{acknowledgements}}

This project was part of SNF Project 310030\_188431. The project was
also supported by the University of Zurich Research Priority Programme
in Global Change and Biodiversity.

\renewcommand\refname{References}
\bibliography{bibliography.bib}


\end{document}
