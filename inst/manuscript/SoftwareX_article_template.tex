%% 
%% Copyright 2007, 2008, 2009 Elsevier Ltd
%% 
%% This file is part of the 'Elsarticle Bundle'.
%% ---------------------------------------------
%% 
%% It may be distributed under the conditions of the LaTeX Project Public
%% License, either version 1.2 of this license or (at your option) any
%% later version.  The latest version of this license is in
%%    http://www.latex-project.org/lppl.txt
%% and version 1.2 or later is part of all distributions of LaTeX
%% version 1999/12/01 or later.
%% 
%% The list of all files belonging to the 'Elsarticle Bundle' is
%% given in the file `manifest.txt'.
%% 

%% Template article for Elsevier's document class `elsarticle'
%% with numbered style bibliographic references
%% SP 2008/03/01

\documentclass[preprint,12pt, a4paper]{elsarticle}

%% Use the option review to obtain double line spacing
%% \documentclass[authoryear,preprint,review,12pt]{elsarticle}

%% For including figures, graphicx.sty has been loaded in
%% elsarticle.cls. If you prefer to use the old commands
%% please give \usepackage{epsfig}

%% The amssymb package provides various useful mathematical symbols
\usepackage{amssymb}
%% The amsthm package provides extended theorem environments
%% \usepackage{amsthm}

%% The lineno packages adds line numbers. Start line numbering with
%% \begin{linenumbers}, end it with \end{linenumbers}. Or switch it on
%% for the whole article with \linenumbers.
\usepackage{lineno}

\usepackage{float}
\restylefloat{table}

\journal{SoftwareX}

\begin{document}

\begin{frontmatter}

%% Title, authors and addresses

%% use the tnoteref command within \title for footnotes;
%% use the tnotetext command for theassociated footnote;
%% use the fnref command within \author or \address for footnotes;
%% use the fntext command for theassociated footnote;
%% use the corref command within \author for corresponding author footnotes;
%% use the cortext command for theassociated footnote;
%% use the ead command for the email address,
%% and the form \ead[url] for the home page:
%% \title{Title\tnoteref{label1}}
%% \tnotetext[label1]{}
%% \author{Name\corref{cor1}\fnref{label2}}
%% \ead{email address}
%% \ead[url]{home page}
%% \fntext[label2]{}
%% \cortext[cor1]{}
%% \address{Address\fnref{label3}}
%% \fntext[label3]{}

\title{Title/Name of your software}

%% use optional labels to link authors explicitly to addresses:
%% \author[label1,label2]{}
%% \address[label1]{}
%% \address[label2]{}

\author{A. Author}

\address{Your institute, some address}

\begin{abstract}
%% Text of abstract 
Ca. 100 words

\end{abstract}

\begin{keyword}
%% keywords here, in the form: keyword \sep keyword
keyword 1 \sep keyword 2 \sep keyword 3

%% PACS codes here, in the form: \PACS code \sep code

%% MSC codes here, in the form: \MSC code \sep code
%% or \MSC[2008] code \sep code (2000 is the default)

\end{keyword}

\end{frontmatter}

\section*{Required Metadata}
\label{}

\section*{Current code version}
\label{}

Ancillary data table required for subversion of the codebase. Kindly replace examples in right column with the correct information about your current code, and leave the left column as it is.

\begin{table}[H]
\begin{tabular}{|l|p{6.5cm}|p{6.5cm}|}
\hline
\textbf{Nr.} & \textbf{Code metadata description} & \textbf{Please fill in this column} \\
\hline
C1 & Current code version & For example v42 \\
\hline
C2 & Permanent link to code/repository used for this code version & For example: $https://github.com/mozart/mozart2$ \\
\hline
C3 & Code Ocean compute capsule & For example: $https://codeocean.com/2017/07/30/neurospeech-colon-an-open-source-software-for-parkinson-apos-s-speech-analysis/code$\\
\hline
C4 & Legal Code License   & List one of the approved licenses \\
\hline
C5 & Code versioning system used & For example svn, git, mercurial, etc. put none if none \\
\hline
C6 & Software code languages, tools, and services used & For example C++, python, r, MPI, OpenCL, etc. \\
\hline
C7 & Compilation requirements, operating environments \& dependencies & \\
\hline
C8 & If available Link to developer documentation/manual & For example: $http://mozart.github.io/documentation/$ \\
\hline
C9 & Support email for questions & \\
\hline
\end{tabular}
\caption{Code metadata (mandatory)}
\label{} 
\end{table}


\linenumbers

%% main text

The permanent link to code/repository or the zip archive should include the following requirements: 

README.txt and LICENSE.txt.

Source code in a src/ directory, not the root of the repository.

Tag corresponding with the version of the software that is reviewed.

Documentation in the repository in a docs/ directory, and/or READMEs, as appropriate.




\section{Motivation and significance}
\label{}

Introduce the scientific background and the motivation for developing the software.

Explain why the software is important, and describe the exact (scientific) problem(s) it solves.

Indicate in what way the software has contributed (or how it will contribute in the future) to the process of scientific discovery; if available, this is to be supported by citing a research paper using the software.

Provide a description of the experimental setting (how does the user use the software?).

Introduce related work in literature (cite or list algorithms used, other software etc.).


\section{Software description}
\label{}

Describe the software in as much as is necessary to establish a vocabulary needed to explain its impact. 

\subsection{Software Architecture}
\label{}

Give a short overview of the overall software architecture; provide a pictorial component overview or similar (if possible). If necessary provide implementation details.

\subsection{Software Functionalities}
\label{}

Present the major functionalities of the software.

\subsection{Sample code snippets analysis (optional)}
\label{}

\section{Illustrative Examples}
\label{}

Provide at least one illustrative example to demonstrate the major functions.

Optional: you may include one explanatory video that will appear next to your article, in the right hand side panel. (Please upload any video as a single supplementary file with your article. Only one MP4 formatted, with 50MB maximum size, video is possible per article. Recommended video dimensions are 640 x 480 at a maximum of 30 frames/second. Prior to submission please test and validate your .mp4 file at $ http://elsevier-apps.sciverse.com/GadgetVideoPodcastPlayerWeb/verification$. This tool will display your video exactly in the same way as it will appear on ScienceDirect.).

\section{Impact}
\label{}

\textbf{This is the main section of the article and the reviewers weight the description here appropriately}

Indicate in what way new research questions can be pursued as a result of the software (if any).

Indicate in what way, and to what extent, the pursuit of existing research questions is improved (if so).

Indicate in what way the software has changed the daily practice of its users (if so).

Indicate how widespread the use of the software is within and outside the intended user group.

Indicate in what way the software is used in commercial settings and/or how it led to the creation of spin-off companies (if so).

\section{Conclusions}
\label{}

Set out the conclusion of this original software publication.

\section{Conflict of Interest}
Please select the appropriate text:

Potential conflict of interest exists:
We wish to draw the attention of the Editor to the following facts, which may be considered as potential conflicts of interest, and to significant financial contributions to this work. The nature of potential conflict of interest is described below: [Describe conflict of interest]

No conflict of interest exists:
We wish to confirm that there are no known conflicts of interest associated with this publication and there has been no significant financial support for this work that could have influenced its outcome.


\section*{Acknowledgements}
\label{}

Optionally thank people and institutes you need to acknowledge. 

%% The Appendices part is started with the command \appendix;
%% appendix sections are then done as normal sections
%% \appendix

%% \section{}
%% \label{}

%% References:
%% If you have bibdatabase file and want bibtex to generate the
%% bibitems, please use
%%
%%  \bibliographystyle{elsarticle-num} 
%%  \bibliography{<your bibdatabase>}

%% else use the following coding to input the bibitems directly in the
%% TeX file.

\begin{thebibliography}{00}


%% \bibitem{label}
%% Text of bibliographic item

\bibitem{}

\end{thebibliography}
Please add the reference to the software repository if DOI for software  is available. 

\section*{Current executable software version}
\label{}

Ancillary data table required for sub version of the executable software: (x.1, x.2 etc.) kindly replace examples in right column with the correct information about your executables, and leave the left column as it is.

\begin{table}[!h]
\begin{tabular}{|l|p{6.5cm}|p{6.5cm}|}
\hline
\textbf{Nr.} & \textbf{(Executable) software metadata description} & \textbf{Please fill in this column} \\
\hline
S1 & Current software version & For example 1.1, 2.4 etc. \\
\hline
S2 & Permanent link to executables of this version  & For example: $https://github.com/combogenomics/$ $DuctApe/releases/tag/DuctApe-0.16.4$ \\
\hline
S3 & Legal Software License & List one of the approved licenses \\
\hline
S4 & Computing platforms/Operating Systems & For example Android, BSD, iOS, Linux, OS X, Microsoft Windows, Unix-like , IBM z/OS, distributed/web based etc. \\
\hline
S5 & Installation requirements \& dependencies & \\
\hline
S6 & If available, link to user manual - if formally published include a reference to the publication in the reference list & For example: $http://mozart.github.io/documentation/$ \\
\hline
S7 & Support email for questions & \\
\hline
\end{tabular}
\caption{Software metadata (optional)}
\label{} 
\end{table}

\end{document}
\endinput
%%
%% End of file `SoftwareX_article_template.tex'.
